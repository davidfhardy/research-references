% \usepackage[utf8]{inputenc} % ignore with LuaLaTeX

% **********************************************************
%                        PACKAGES
% **********************************************************

\usepackage[letterpaper, hmargin=1.5cm, top=3cm, bottom=2cm, headheight=17pt, headsep=1cm]{geometry}
\usepackage{blindtext}  % for creating random paragraphs
\usepackage{fancyhdr}   % for header and footer
\usepackage{titlesec}   % formats section titles 
\usepackage[T1]{fontenc}
\usepackage{fontspec}
\usepackage[bitstream-charter]{mathdesign} % alternative nice font 
\usepackage{amsmath}    % math package
%\usepackage{amssymb}   
\usepackage{amsthm}     % math theorems
\usepackage{empheq}     % emphasizing/boxing equations
\usepackage{textcomp}   % load this before gensym to avoid warnings
\usepackage{gensymb}    % general symbols
\usepackage{enumitem}   % formatting enumerate and itemize environments
\usepackage{graphicx}   % for figures/images
\usepackage{longtable}  % for allowing tables to span onto another page
\usepackage{wrapfig}    % 
\usepackage{float}      % making figures floats with [H]
\usepackage{multicol, multirow} % to make multiple columns/rows
\usepackage{caption}    % formatting figure/table captions
\usepackage{hyperref}   % formmatting hyperlinks and other links
\usepackage{setspace}   
\usepackage{siunitx}    
\usepackage[framemethod=TikZ]{mdframed} 
\usepackage{thmtools}
\usepackage[nottoc,notlot,notlof]{tocbibind}    
\usepackage[toc, titletoc]{appendix}
\usepackage{listings}
%\usepackage{subfigure}
\usepackage{subcaption}
\usepackage{parskip}

\usepackage{tcolorbox}
\tcbuselibrary{skins,xparse,breakable}

\usepackage[customcolors]{hf-tikz}
\usepackage{chngcntr} % change counter
\usepackage{esint} % for drawing [closed] integral symbols, like \oiint
\usepackage{titling} % for title page
\usepackage{steinmetz} % for angle symbols with phasors
\usepackage{lastpage}

\usepackage[style=ieee]{biblatex}
\addbibresource{references.bib} %Import the bibliography file


% for calligraphy letters. 
% e.g. {\calligra E}
%\usepackage{calligra} 
%\usepackage{mathrsfs}
\usepackage[scr=boondoxo]{mathalpha}
%\usepackage[scr=boondox, cal=rsfs]{mathalpha}

\usepackage{minted}

% *************************************************************
%  FONTS
% *************************************************************


% *************************************************************
%  COLOURS
% *************************************************************
\definecolor{navyblue}{rgb}{0.0, 0.0, 0.5}
\definecolor{persianblue}{rgb}{0.11, 0.22, 0.73}
\definecolor{pumpkin}{rgb}{1.0, 0.46, 0.09}
\definecolor{harvardcrimson}{rgb}{0.79, 0.0, 0.09}
\definecolor{forestgreen}{rgb}{0.13, 0.55, 0.13}
\definecolor{blue-violet}{rgb}{0.54, 0.17, 0.89}

\definecolor{figColour}{RGB}{65,105,225}
\definecolor{itColour}{RGB}{220,20,60}
%\definecolor{eqnRef}{RGB}{0, 155, 93} 
\definecolor{eqnRef}{rgb}{0.13, 0.55, 0.13}
\definecolor{violetBlue}{HTML}{551ADE}


\definecolor{codegreen}{rgb}{0,0.6,0}
\definecolor{codegray}{rgb}{0.5,0.5,0.5}
\definecolor{codepurple}{rgb}{0.58,0,0.82}
\definecolor{backcolour}{rgb}{0.95,0.95,0.92}


% *************************************************************
%  MATH DEFINITIONS 
% *************************************************************
\renewcommand{\vec}[1]{\boldsymbol{\mathrm{#1}}}
\newcommand{\uvec}[1]{\boldsymbol{{\mathrm{\hat{#1}}}}}
\renewcommand{\Re}{\boldsymbol{\mathscr{R} \hspace{-3pt} \mathscr{e}}}
\renewcommand{\Im}{\boldsymbol{\mathscr{I} \hspace{-3pt} \mathscr{m}}}
\newcommand{\phasor}[1]{\widetilde{#1}}
\newcommand{\vecphasor}[1]{\widetilde{\mathbf{#1}}}
\newcommand{\timevec}[1]{\bar{\mathscr{#1}}}
\newcommand{\divergence}{\nabla\cdot}
\newcommand{\curl}{\nabla\times}


\newcommand{\rthetaphi}{(r,\theta,\phi)}
\newcommand{\thetaphi}{(\theta,\phi)}
\newcommand{\Efield}{\vec{E}(\vec{r})}
\newcommand{\Hfield}{\vec{H}(\vec{r})}

\newcommand{\txtin}{\mathrm{in}}
\newcommand{\txtinc}{\mathrm{inc}}
\newcommand{\txtavg}{\mathrm{avg}}
\newcommand{\txtrad}{\mathrm{rad}}
\newcommand{\txtloss}{\mathrm{loss}}
\newcommand{\txtrefl}{\mathrm{refl}}
\newcommand{\txtant}{\mathrm{ant}}
\newcommand{\txtco}{\mathrm{co}}
\newcommand{\txtcr}{\mathrm{cr}}
\newcommand{\txtdb}{\mathrm{dB}}
\newcommand{\txtpo}{\mathrm{PO}}
\newcommand{\txtmax}{\mathrm{max}}
\newcommand{\txtmin}{\mathrm{min}}

% *************************************************************
%  CUSTOM COMMANDS
% *************************************************************
\newcommand{\coltextit}[1]{\textbf{\textit{\color{itColour}{#1}}}}
\newcommand{\paperitem}[1]{\textit{\cite{#1} \citefield{#1}{title} (\citefield{#1}{year})}}

% *************************************************************
% MY COUNTERS
% *************************************************************
\renewcommand{\thefigure}{\arabic{figure}}

\newcommand*{\refeqn}[1]{%
  \begingroup
    \hypersetup{
      linkcolor=eqnRef,
      linkbordercolor=eqnRef,
    }%
    \textcolor{eqnRef}{Eq.\ (\ref{#1})}%
  \endgroup
}

% *************************************************************
% HYPERLINK SETUP
% *************************************************************
\urlstyle{texttt}
\hypersetup{
    colorlinks=true,
    linkcolor=black,
    filecolor=magenta,  
    citecolor=black,
    urlcolor=black
}
   
% *************************************************************
% HEADERS AND FOOTERS
% *************************************************************
\pagestyle{fancy}
\renewcommand{\headrulewidth}{0.5pt}
\fancyhf{}
\renewcommand{\chaptermark}[1]{\markboth{{#1}}{}}
\lhead{\textsl{\chaptertitlename \ \thechapter. \leftmark}}
\rhead{\thepage \ of \pageref{LastPage}}
\cfoot{}

% *************************************************************
% SECTIONS/CHAPTERS
% *************************************************************

% Sections
\titleformat{\section}
   {\Large\bfseries\sffamily}{\thesection}{0.5em}{}
\titleformat{\subsection}
   {\large\bfseries\sffamily}{\thesubsection}{0.5em}{}
\titleformat{\subsubsection}
   {\normalsize\bfseries}{\thesubsubsection}{0.5em}{}
%\newcommand{\sectionbreak}{\clearpage} % adds new page for each section
% Chapter
% \titleformat{\chapter}[frame]
% {\normalfont}
% {\filcenter
% \Large
% \enspace \bfseries \textcolor{harvardcrimson}{\textit{\chaptertitlename\ \thechapter}}
% }
% {8pt}
% {\huge\bfseries\filcenter}
\renewcommand{\chaptertitlename}{Section}

\titleformat{\chapter}[display]
{\bfseries\Large\sffamily}
{\filleft\MakeUppercase{\chaptertitlename} \Huge\thechapter}
{0.5cm}
{\titlerule[2pt]\LARGE
\vspace{0.5cm}%
\centering}
[\vspace{0.5cm}%
{\titlerule[2pt]} ]

\titlespacing*{\chapter}{0pt}{-30pt}{20pt}


% *************************************************************
% PARAGRAPHS AND LISTS/ENUMERATE
% *************************************************************
\setlist{partopsep=0pt, topsep=-5pt, itemsep=-7pt, after=\vspace{0.25cm}} % Customize spacing around and inside lists
\setlist[itemize]{label=\footnotesize$\blacksquare$}

% *************************************************************
% FIGURES AND TABLES
% *************************************************************

% Figure/Table counters
\counterwithin{figure}{chapter}
\counterwithin{table}{chapter}

% Figures caption formatting
\captionsetup{%
  figurename=Fig.,
  tablename=Tab., 
  labelfont={bf,color=black}, 
  font={small, sl}
}

% Tables
\def\arraystretch{1.5}

% *************************************************************
% EQUATIONS
% *************************************************************

\allowdisplaybreaks
\counterwithin{equation}{chapter}

% *************************************************************
% MDFRAMED
% *************************************************************
\mdfsetup{leftmargin=0, rightmargin=0, innertopmargin=3pt, innerbottommargin=3pt}
\mdfdefinestyle{equation}{backgroundcolor=green!10, linecolor=green!10}
\mdfdefinestyle{equations}{backgroundcolor=green!10, linecolor=green!10, innertopmargin=-10pt}

% Equation boxes
\hfsetfillcolor{green!10}
\hfsetbordercolor{green!50!black}

\definecolor{eqnGreen}{rgb}{.8, 1, .8}
\newcommand*\eqnGreenBox[1]{%
\colorbox{eqnGreen}{\hspace{1em}#1\hspace{1em}}}
\setlength{\fboxsep}{0.5em}

\theoremstyle{definition}

\declaretheoremstyle[
    headfont=\bfseries\sffamily\color{persianblue!90!black}, bodyfont=\normalfont,
    mdframed={
        linewidth=2pt,
        rightline=false, topline=false, bottomline=false,
        linecolor=persianblue, backgroundcolor=persianblue!5,
    }
]{examplebox}

\declaretheoremstyle[
    headfont=\bfseries\sffamily\color{pumpkin!90!black}, bodyfont=\normalfont,
    mdframed={
        linewidth=2pt,
        rightline=false, topline=false, bottomline=false,
        linecolor=pumpkin, backgroundcolor=pumpkin!5,
    }
]{notebox}

\declaretheoremstyle[
    headfont=\bfseries\sffamily\color{blue-violet!90!black}, bodyfont=\normalfont,
    mdframed={
        linewidth=2pt,
        rightline=false, topline=false, bottomline=false,
        linecolor=blue-violet, backgroundcolor=blue-violet!5,
    }
]{proofbox}

\declaretheorem[style=examplebox, numberwithin=chapter, name=Example]{example}
\declaretheorem[style=notebox, numbered=no, name=Note]{note}
\declaretheorem[style=proofbox, numberwithin=chapter, name=Proof]{proofBox}







