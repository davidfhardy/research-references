\chapter{Metasurfaces} 

\paperitem{Lijun2024_HuygensTransmitarray}
\begin{itemize}
    \item Yet to read...
\end{itemize}

% =================================================
% Optically Transparent Reflectarrays 
% =================================================
\section{Optically-Transparent Metasurfaces}

\paperitem{Green2019_OpticallyTransparentAntennas} 
\begin{itemize}
    \item Good information on the need/relevance for transparent wireless devices
    \item Shows history and development of transparent antennas
    \item Transparent antennas were developed initially for window-embedded applications
    \item An emerging application is for satellite communications
    \item Two main categories for transparent antennas: (1) Meshed. (2) Thin-film
    \item Meshed antennas: antennas made with regular conductors (like copper) but with patterns and holes to allow light to pass through. The performance (gain, return loss, efficiency) is usually smaller than its counterpart. Transparency can be calculated as $T = (A_\text{solid} - A_\text{mesh})/A_\text{solid}$
    \item Thin-film antennas: these use transparent conductive oxides (TCOs), polymers, conductive inks, and things along these lines. ITO is the most popular TCO used but indium shortages are expected in the near future. 
    \item For TCOs, the transparency is greater with thinner films, but the conductivity is better for thicker films, so there is a tradeoff. 
\end{itemize}

\paperitem{Chen2022_FineLineMetals}
\begin{itemize}
    \item Yet to read...
\end{itemize}

\paperitem{Heo2024_OpticallyTransparentMSarray}
\begin{itemize}
    \item Yet to read... 
\end{itemize}